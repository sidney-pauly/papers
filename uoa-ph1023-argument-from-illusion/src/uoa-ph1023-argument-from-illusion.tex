\documentclass[fleqn,14pt]{article}

\usepackage[letterpaper,margin=0.75in]{geometry}

\usepackage{amsmath}
\usepackage{booktabs}
\usepackage{graphicx}
\usepackage{listings}
\usepackage{fancyhdr}
\usepackage{standalone}
\usepackage{float}
\usepackage{hyperref}
\usepackage{biblatex} %Imports biblatex package

% Bibliography
\addbibresource{all.bib}

% \include{data/reaction-time.csv}

\setlength{\parindent}{1.4em}

\pagestyle{fancy}


\begin{document}

\lstset{
  language=Python,
  basicstyle=\small,          % print whole listing small
  keywordstyle=\bfseries,
  identifierstyle=,           % nothing happens
  commentstyle=,              % white comments
  stringstyle=\ttfamily,      % typewriter type for strings
  showstringspaces=false,     % no special string spaces
  numbers=left,
  numberstyle=\tiny,
  numbersep=5pt,
  frame=tb,
}

\title{What is the argument from illusion or hallucination, and how might one criticize it?}
\date{}



\author{Sidney Pauly}
\def\theuoastudentid{52104132}

\makeatletter

\let\thetitle\@title
\let\theauthor\@author
\let\thedate\@date


\makeatother




\fancyhf{}
\fancyhead[L]{Name: \theauthor}
\fancyhead[R]{ID: \theuoastudentid}


% \maketitle

\begin{titlepage}
  \begin{center}
    \Large
    \textbf{\thetitle}
        
    \vspace{0.4cm}
    \large
    PH1023 - Essay 2
        
    \vspace{0.4cm}
    \textbf{\theauthor}\\
    \textbf{\theuoastudentid}

       
    \vfill


    University of Aberdeen\\
    Scotland\\
    UK\\
    \thedate
    \vspace{0.4cm}
    \url{https://github.com/sidney-pauly/papers}
  \end{center}
\end{titlepage}


% \section{Introduction}

The argument from illusion or hallucination tells us that our perception of reality is not direct. There are
different theories on the effect this has on our knowledge and our understanding of the world. Some more extreme theories
only leaves the conclusion that we can not truly know anything about the world, leaving only complete
skepticism as an option. In this essay, I will show that no matter what further conclusion one draws from the
argument from illusion or hallucination, it should not lead us to believe that we cannot have any insight
into what reality is about.\\

% As seen from the argument from illusion, reality is always perceived indirectly through sensory input. In
% response some arguments have been put forward stating, that there is an underlying reality which produces or
% invokes those senses, which itself cannot be observed. Others state that because of the indirect nature
% of our senses it might be that the "real" reality could be totally different from what we perceive. In some
% versions of that argument those realities, are as well inaccessible to sensory enquirers.
% The proposal of this essay being that those theories those should be disregarded when discussing human
% knowledge. While the theories might very well be valid or even true their
% results tell us nothing about reality.\\

% \section{Explaining the argument from illusion}
The argument form illusion or hallucination goes as follows\cite[p.295]{sosa}:

\begin{enumerate}
  \item Human perception can be deceived either when under illusion or
  hallucination. In other words, objects might appear to have properties that
  they do not.
  \item In such cases of illusion or hallucination, one are directly aware of an
  object with certain properties, which steams from an object that in no way
  has to resemble the imagined one.
  \item If this is the case, one is not directly aware of the actual object 
  \item Therefore, even when not hallucinating or under an illusion, one is not directly aware
  of the actual object
\end{enumerate}
From this argument, all sorts of further inquiries about reality are made. No matter what they
come up with,
they all need to solve our connection to "base" reality and how we can, if ever, know anything about it.\\

% \section{Responses to the argument}
% \subsection{Indirect realism}
One response to the argument is indirect realism. Postulated is that one should actually accept the indirectness
of our perception of reality. All knowledge we can hope to acquire about reality needs to originate in perception.
In the base version of this idea, this means that we cannot really know anything about reality\cite[p.71ff]{pritchard}.
The sensory data we
get\footnote{The theory is sometimes also called sense-datum theory} could be entirely different from how the
world is. Variations on the idea try to rectify this inability to know anything about
the world. One is disjunctivism, which separates the truth value of a belief from the perception.
Sosa criticizes the idea by pointing out that two individuals could have exactly the same sensory experience,
one experiencing the real thing, the other hallucinating, while only one of them being right
\cite[p.299ff]{sosa}.
He thinks this is problematic, as for an individual, relying on sensory
input alone (as all humans do), there is no way to know who is right. Sosa himself, therefore, argues for a
slightly different view. He gets rid of the idea that concrete objects themselves cause our 
perception\cite[p.301ff]{sosa}. He argues that the properties of an object are fundamental instead.
Differentiating the veridical case from the deceived is that in the latter, there are always some properties missing.
As an example, Sosa takes a red post-box. Both in the hallucinatory and real case, the subject experiences 
the object to have the same quality (or property), e.g. its redness. What makes the real post-box real
is that it has some more properties that the hallucinatory one lacks. Sosa does not make clear what those properties are
though, they might, for example, be other relations to other objects. Like another person experiencing the same
redness of the same post-box. This property could not be present in a simple case of hallucination or deception\\
\\

% \subsection{Idealism}
Another way to solve the problem is idealism. Rather than explaining how our senses are connected
to reality, the idea of reality is abandoned altogether. Perceptions are instead made fundamental.
This solves how we can truly know anything by just stating that what we perceive
to be real actually is real. On the other hand, it disconnects us from any tangible or coherent reality.
Under this view, it is not even guaranteed that things exist if we do not look at
them \cite[p.73ff]{pritchard}.\\
A modification of the idea which rectifies the issue is Kant's transcendental idealism.
In his view, reality is also dependent on perception. Differentiating his form of idealism from the "normal"
version is that he thinks there is an objective reality\cite{otfried}. Kant proposes that this objective
reality can be reached through reason. He calls these fundamental things that constitute the objective
reality "das Ding an sich" (The thing itself). In the context of perception, this means nothing is beyond our
experience as one can always arrive there through reason. \\
\\


% \section{Core argument}
With this in mind it can be examined, why no matter what view one subscribes to, it is impossible for reality
to be more than what can be found out through our senses.
Let us suppose our reality was not fundamental
and what is commonly though of as the base constituents of our reality \footnote{e.g. the fundamental
particles, or any smaller, yet undiscovered physical stuff}, are not the base level of reality.
To aid imagination, a perfect matrix style simulation can be taken as an example. In the simulation,
the inhabitants would have no way of detecting (even in theory) that their reality is not fundamental.
Everything they look at seems to be part of their world and not be governed by anything else. This means
there is no link, no communication between the layer
the simulation runs on and them. To refer to the context of the film, there is no Morpheus appearing,
no glitch and no possibility to wake up in a vat with tubes connected to oneself. There is not even a 
theoretical possibility for any of these things to happen. Such a reality would be similar to what is
described by many philosophers that argue we can never truly know about reality because our perception
is fundamentally limited.\\
What does this setup tell us about perception??\\
It is about the connection between the layers. If there is no way for the simulating layer to impact 
the simulated one, the former has no relevance for the latter. With such a setup the simulated
individuals have nothing to gain from believing in the simulated world, even if it exists. It would
add nothing to their understanding of reality knowing about it. Everything they experience can be explained
by stuff perceivable by them. Because there are no effects from the simulating layer, no rules,
no goings-on in the fundamental reality, affect the perceivable one.
As a result, they needn't be included in any theories explanting the simulated one. \\
This fact alone is not enough to explain why, for the simulated beings, there is no difference if a base-layer
exists.
One could bring up that there might be phenomenons which the people living in the lesser reality could never
grasp because they do not know about the actual processes governing them. The key to why this does not disprove
the idea is reason. As an illustration, imagine a solid block appearing in a town square of the
simulated reality. The actual reason for it to appear is that someone from the simulating layer
was pushing a button. Now this phenomenon would not be explainable through the physics of the simulated world.
Nowhere in their description of the world, an explanation
exists for why the block appears. It seems like in order to explain the phenomena they need insight into
a process which they could not possibly perceive. \\
Why is this not the case then? \\
The reason is that it would break the perfect separation of the layers. By introducing such
a connection, the inhabitants of the simulated world now indeed have a perceptual connection to the simulating
one. They can, by common sense, infer that there must be e.g. a being (they might call it god) that
makes the block appear. Also, they might be able to mess with the phenomenon and therefore start communicating
back to the simulating layer. Maybe they figured out that they could stand in the spot where
the block usually appeared and it would not do so. They might reason that the outer being pushing the button 
had empathy and would not push it for that reason. Through that interaction, they could gather further information
about the outer world, further breaking the connection. \\

Kant himself pointed this out in his book "Critic of pure reason". He states, this fact by telling
us that stuff lying outside our perception and reason is illegitimate: "this perspective
denies us any insight into the 'inner' nature of things. But if such complaints merely signify that 'we cannot conceive by pure understanding
what the things which appear to us may be in themselves, they are
entirely illegitimate and unreasonable' since human beings cannot
know anything at all 'without the senses'"\cite[p.49]{otfried} \\

Worth mentioning are two caveats to the notion. For one, we need to accept that human reason,
is not in some fundamental way limited. If reasonable
beings fail to imagine how reality actually is, then it does not help that everything
can be picked up through the senses. An answer to this is fundamentally out of reach, because
philosophers themselves are such potentially limited reasonable beings and can thus never hope
to know about a way of reasoning that goes beyond their own abilities. The only exception being
perceptional evidence of such reasoning, which would lift the phenomenon into the perceptional realm,
thereby making it accessible to inquiry again. \\

A second more serious problem is that of proof. Only because we can imagine, through reason, 
how the fundamental reality could be, it might not be possible to decide, just through pure reason, which
answer is the correct or best one. Therefore, some proof would be required to decide on one version.
Even if the base realities phenomena bleed over into observable reality, this does not ensure that the
observable effects are enough to prove the fundamental reality being one way or another. \\

While the argument presented does not tell us, what belief is justified, we can at least be certain,
that in theory, a complete understanding of reality is possible. This helps to fend off skepticism and
makes supporting some ideas easier. Of the arguments presented on the matter, Kant's transcendental idealism
fits best, though some forms of realism are also well suited.

\printbibliography 

\end{document}