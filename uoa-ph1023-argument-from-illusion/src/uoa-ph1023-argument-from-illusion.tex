\documentclass[fleqn,14pt]{article}

\usepackage[letterpaper,margin=0.75in]{geometry}

\usepackage{amsmath}
\usepackage{booktabs}
\usepackage{graphicx}
\usepackage{listings}
\usepackage{fancyhdr}
\usepackage{standalone}
\usepackage{float}
\usepackage{hyperref}
\usepackage{biblatex} %Imports biblatex package

% Bibliography
\addbibresource{all.bib}

% \include{data/reaction-time.csv}

\setlength{\parindent}{1.4em}

\pagestyle{fancy}


\begin{document}

\lstset{
  language=Python,
  basicstyle=\small,          % print whole listing small
  keywordstyle=\bfseries,
  identifierstyle=,           % nothing happens
  commentstyle=,              % white comments
  stringstyle=\ttfamily,      % typewriter type for strings
  showstringspaces=false,     % no special string spaces
  numbers=left,
  numberstyle=\tiny,
  numbersep=5pt,
  frame=tb,
}

\title{What is the argument from illusion or hallucination, and how might one criticize it?}
\date{}



\author{Sidney Pauly}

\makeatletter

\let\thetitle\@title
\let\theauthor\@author
\let\thedate\@date
\def\theuoastudentid{52104132}

\makeatother




\fancyhf{}
\fancyhead[L]{Name: \theauthor}
\fancyhead[R]{ID: \theuoastudentid}


% \maketitle

\begin{titlepage}
  \begin{center}
    \Large
    \textbf{\thetitle}
        
    \vspace{0.4cm}
    \large
    PH1023 - Essay 2
        
    \vspace{0.4cm}
    \textbf{\theauthor}\\
    \textbf{\theuoastudentid}

       
    \vfill


    University of Aberdeen\\
    Scotland\\
    UK\\
    \thedate
    \vspace{0.4cm}
    \url{https://github.com/sidney-pauly/papers}
  \end{center}
\end{titlepage}


% \section{Introduction}

The argument from illusion or hallucination tells us that our perception of reality is not direct. There are
different theories on what effect his has on our knowledge on our world. With some more extreme theories
only leaving the conclusion, that we can't truly know anything about the world, leaving only complete
skepticism as an option. In this essay I'll show that no matter what further conclusion one draws from the
argument from illusion or hallucination, it should not lead us to believe that we cannot have any insight
into what reality really is about.

% As seen from the argument from illusion, reality is always perceived indirectly through sensory input. In
% response some arguments have been put forward stating, that there is an underlying reality which produces or
% invokes those senses, which itself cannot be observed. Others state that because of the indirect nature
% of our senses it might be that the "real" reality could be totally different from what we perceive. In some
% versions of that argument those realities, are as well inaccessible to sensory enquirers.
% The proposal of this essay being that those theories those should be disregarded when discussing human
% knowledge. While the theories might very well be valid or even true their
% results tell us nothing about reality.\\

% \section{Explaining the argument from illusion}
The argument form illusion or hallucination goes as follows\cite[p.295]{sosa}:

\begin{enumerate}
  \item Human perception can be deceived either when there is illusion or
  hallucination. In other words objects might appear to have properties, that
  they don't.
  \item In such cases of illusion or hallucination we are directly aware of an
  object with certain properties, which steams from an object that in now way
  has to resemble the imagined one.
  \item If this is the case, we are not directly aware of the actual object 
  \item Therefore, event when not hallucinating or under illusion, we are not directly aware
  of the actual object
\end{enumerate}
From this argument different, all sorts of further enquiries about reality are made. All of them need to
present an answer on how we are connected to "base" reality and how we can if ever know anything about it.\\
\\

% \section{Responses to the argument}
% \subsection{Indirect realism}
One response to the argument is indirect realism. Postulated is that one should actually accept the indirectness
of our perception of reality. All that we know of the world is the sensory perception in invokes in us. In the
base version of this idea this means that we cannot really know anything about reality\cite[p.71ff]{pritchard} The sensory data we
get\footnote{The theory is sometimes also called sense-datum theory} could be entirely different form how the
world actually is. Some variations on the idea try to rectify, this inability to truly know anything about
the world. One of them is disjunctivism, which separates the truth value of a believe from the perception. 
Sosa criticizes the idea, by pointing out that two individuals could have the exact same sensory experience,
one experiencing the real thing, the other hallucinating, while only one of them being right
\cite[p.299ff]{sosa}. He thinks this is problematic, as there is no way for an individual, relying on sensory
input (as all humans do), to know, who is right. Sosa himself therefore argues for a slightly different
view in which he gets rid of the notion where concrete objects themselves cause our 
perception\cite[p.301ff]{sosa}. Instead he
argues that the properties of the objects are fundamental. What in his view differentiates the hallucinatory
case from the real one is that in the former there are always some properties missing. As an example he gives
is a red post-box, where in both the hallucinatory case and the real one, subject experiences the same quality
(or property), e.g. the redness. What makes the real post-box real is that it has some more properties that
the hallucinatory one lacks. Sosa does make clear what those properties are, though, they could be other relations
to other object, e.g. another person experiencing the same redness of the same post-box, which wouldn't be the
case if the post-box was stemming from hallucination.\\
\\

% \subsection{Idealism}
Another way to solve the problem is idealism. Instead of trying to explain how our senses are connected
to reality, instead the idea of reality is abandoned altogether. Instead perceptions are made fundamental.
This solves the problem, of how we can truly know something, by just stating that what we perceive
to be real, actually being real. On the other hand it disconnects us from any tangible or coherent reality.
Under this view it is not even guaranteed, that things exists if we look away 
from them \cite[p.73ff]{pritchard}.\\
A modified version of the idea which rectifies the issue is Kant's transcendental idealism.
In his view reality is also not independent from 
perception. Differentiating, it from "normal" idealism in the context of the presented argument, is
that he thinks there is an objective reality\cite{otfried}. Kant proposes that this objective reality can be reached through
reason. He calls these fundamental things, that constitute the objective
reality "Das Ding an sich" (The thing itself). In the context of perception this means nothing is beyond our
experience as one can always arrive there through reason. \\
\\


% \section{Core argument}
With this in mind it can be examined, why no matter what view one subscribes to, it is impossible for reality
to be more than what can be found out through our senses.
Lets suppose our reality was not fundamental
and what is commonly though of as the base constituents of our reality \footnote{e.g. the fundamental
particles, or any smaller, yet
undiscovered physical things}, are in fact not the base level of reality. To aide with imagination, a perfect
matrix style simulation can be taken as an example. In this simulation, the inhabitants have no way of detecting
(even in theory) that their reality is not fundamental. Everything they look at seems to be part of their
world and not be governed by anything else. This means there is no link, no communication between the layer
the simulation runs on and them. To speak in the pictures of the film, there is no Morpheus appearing,
no glitch and no possibility to wake up in a vat with tubes connected to oneself. More, there not even a
theoretical possibility, for any of these things to happen. Such a reality would be similar to what is
described by many philosophers that argue we can never truly know about reality, because our perception
is fundamentally limited.\\
Now what is the problem with such a setup?\\
It's about the connection between the layers. If there is no way for the simulating layer to impact the
the simulated one, the former has in fact no relevance for the later. In such a case the simulated
individuals have nothing to gain from believing in the simulated world, even if it in fact exists. It would
add nothing to their reality knowing about it, as everything they experience can be explained by stuff
making up their own perceivable reality. Because there are no effects from the simulating layer, no rules,
no going ons in
the fundamental reality, affecting the subjective one they needn't be included in any theories explanting
the simulated one. \\
This fact alone is not enough to explain why, for the simulated beings, the two cases are in fact indifferent.
One could bring up that there might be phenomenons which the people living in the lesser reality could never
grasp because they don't know about the actual processes governing them. The key to why this does not disprove
the idea is reason. Again to illustrate such an example one could imagine that in the town square of the
simulated reality block of metal would appear every so often. The actual reason for it to appear would
be that someone from the simulating layer was pushing a button. Now this phenomenon would not be explainable
through the physics of the simulated world. Nowhere in their description of the world, an explanation
exists for why the block appears. It seems like in order to explain there world they need insight into
a process which they couldn't possible perceive by their own senses. \\
Why is this not the case then? \\
The reason is that it would brake the perfection of the separation of the two layers, by introducing such
a connection, the inhabitants of the simulated world now indeed have a perceptual connection to the simulating
one. They can, by common sense, infer, that there must be e.g. a being (they might call it god, or whatever), that
makes the block appear. Also they could start to mess with the phenomenon and therefore start communicating
back to the simulating layer. What for example if they figured out that they could stand in the spot where
the block usually appeared and it would not do so because the outer being pushing the button had empathy,
and would not push it for that reason. Now through that interaction they could gather further information
about this outer world further breaking the connection. \\
\\
Kant himself pointed this out in his book "Critic of pure reason" where he wrote that "this perspective
denies us any insight into the 'inner' nature of things. But if such com-
plaints merely signify that 'we cannot conceive by pure understanding
what the things which appear to us may be in themselves, they are
entirely illegitimate and unreasonable' since human beings cannot
know anything at all 'without the senses'"\cite[p.49]{otfried} \\
\\
Worth mentitoning are two caveates to the notion. For one we need a accept that human (or indeed any
reasonable being, as Kant proposes) reason, is not in some fundamental way limited. If reasonable
beings in general simply fail to imagine how reality actually is then the it does not help that everything
can be picked up by the senses in theory. An answer to this is fundamentally out of reach simply because
philosophers themselves are such reasonable beings and can thus never hope to know about a way of reasoning
that goes beyond there own abilities. The only exception being perceptional evidence of such, wich intern
would lift the phenomenon into the perceptional realm and thereby making it accessible to enquiry again.\\

A second more serious problem is that of prove. Only because we can imagine, through reason, 
how the fundamental reality could be, it might not be possible to decide, just through pure reason, which
if that answer is the correct or best one. Therefore some prove would be required, to decide on one version.
Even if the base realities phenomena bleed over into observable reality, this does not ensure that the
observable effects are enough to prove the fundamental reality being one way or another. \\
\\

While the argument presented does not tell us, what in fact is justified belief, we can at leas be certain,
that in theory, a complete understanding of the reality is possible. This helps to fend of skepticism and
makes supporting some ideas easier. Of the arguments presented on the matter, Kant's transcendental idealism
fits best, though some forms of realism are also well suited.


\printbibliography 

\end{document}