\documentclass[14pt]{article}

\usepackage[letterpaper,margin=0.75in]{geometry}

% \usepackage{amsmath}
\usepackage{booktabs}
\usepackage{fancyhdr}
\usepackage{standalone}
\usepackage{float}
\usepackage{hyperref}
% \usepackage{biblatex} %Imports biblatex package
\usepackage{amssymb} % For item list in premise conclusion style

% Bibliography
\usepackage{csquotes}
\usepackage[style=mla]{biblatex}
\addbibresource{all.bib}

% \include{data/reaction-time.csv}

\setlength{\parindent}{1.4em}

\pagestyle{fancy}

\begin{document}


\title{Critical response A.J. Ayer's essay "Freedom and Necessity"}
\date{}



\author{Sidney Pauly}

\makeatletter

\let\thetitle\@title
\let\theauthor\@author
\let\thedate\@date
\def\theuoastudentid{52104132}

\makeatother




\fancyhf{}
\fancyhead[L]{Name: \theauthor}
% \fancyhead[C]{}
\fancyhead[R]{ID: \theuoastudentid}


% \maketitle

\begin{titlepage}
  \begin{center}
    \Large
    \textbf{\thetitle}
        
    \vspace{0.4cm}
    \large
    PH201B - Essay 1
        
    \vspace{0.4cm}
    \textbf{\theauthor}\\
    \textbf{\theuoastudentid}

    \vspace{2cm}
    \textbf{Abstract}

  \end{center}

  In his essay “Freedom and Necessity” A. J. Ayer argues for his compatibilist view.
  Furthermore he makes some remarks as to why indeterminism (true randomness) is not a solution either
  as it would undermine moral responsibility. This essay will focus on summarizing Ayer's compatibilist argument itself,
  and explain where it falls short.

  \vfill

  \begin{center}

    University of Aberdeen\\
    Scotland\\
    UK\\
    \thedate
    \vspace{0.4cm}
    \url{https://github.com/sidney-pauly/papers}
  \end{center}
\end{titlepage}


To understand Ayers argument it makes sense to briefly revisit the core argument as to why free will and determinism are incompatible:

\begin{itemize}
  \item[1.] Determinism means that all future world states are determined only be the current state of the world and causal laws.
  \item[2.] To have free will means to be able act otherwise.
  \item[3.] If the future and therefore all actions are already determined than there is no way to act otherwise.
  \item[$\therefore$] Therefore determinism and free will are incompatible.
\end{itemize}

Ayer takes issue with premise 2. of the argument layed out above and what it implies for premise
3. He argues that we should understand what free will means by looking at it's ordinary usage.
In ordinary language someone is assumed to not act freely if their action is constraint.
Ayer gives multiple examples for various types of constraints.
They could be external to the person, like someone pointing a gun threatening to kill them unless they perform
some action. They could also be internal like having a certain mental condition preventing someone to act
based on their own choices and rather being forced to perform certain things (e.g. kleptomania, where a person
cannot prevent themselves stealing things, regardless of how much they want to).
By contrast acting under free will thereby means to act without any constraint. By extension
Ayer essentially argues that having the ability to do otherwise should be understood as not being constraint
to act a certain way.\\
\\
The obvious objection, which Ayer himself acknowledges,
is that determinism could be considered as such a constraint.
In his own words: “I do not have the feeling of constraint I have when a pistol
is manifestly pointed at my head; but the chains of causation by which I am bound are no less
effective for being invisible”\autocite[145]{Ayer}.
Ayer counters by arguing that we need to qualify what to consider as constraints.
Again drawing from ordinary usage, an action is only regarded
as not free if it has specific causes, specifically ones that constrain the action.
Just because someone says they ate because they where hungry we don't see their
decision on wether they should have a meal as being constraint.
Ayer argues that as causes that explain what happened but do not constrain the resulting action.
Ayer further describes Determinism under this model as only requiring causes to be sufficient or necessary.
I.e. they are only required to offer some causal explanation for what happened.\\
\\
So what is the problem with Ayers compatibilist approach?
While most of his argument seems to be sound and not build
on further assumption, he brings in a new account of
determinism which he poorly justifies. In essence the problem
is that determinism does not (in the way it is ordinarily understood),
mean that events have sufficient \textbf{or} necessary causes rather it means that events have both sufficient
\textbf{and} necessary causes. Indeed R. C. Perry comes to the very same 
conclusion in his discussion piece on Ayer's essay:
“And accepting this law [the law of universal causation], it was not [...]
A as a necessary or sufficient condition of B, but rather A as both a
necessary and sufficient condition for B”\autocite[229]{Perry}.
In his article Perry goes on to explain how Ayer is indeed holding
this exact account of determinism while also holding a contrary one where
causes could be either necessary or sufficient. He shows that Ayers argument
does not hold up as he does not offer a way to reconcile them. \\
\\
Regardless of the essays further problems, to dismiss Ayer's compatibilist
argument it is sufficient
to show how it is inconsistent within itself and how said “or” premise
does not hold with the way Ayer implies determinism to work.
\\
This indeed follows directly from premise 1. stated in the introductory argument defining
determinism\footnote{Ayer himself seems to imply this view on determinism when he talks about
"human behavior [...] entirely governed by causal laws"\autocite[139]{Ayer}}.
For an event to be determined means that said event will happen just by
virtue of the current world state and the causal laws. This
means the previous world state is both a sufficient and necessary
cause for the event to happen, which is best illustrated through an example.\\
Suppose there is a sufficiently large and sturdy object traveling at a high
velocity towards a pain of glass. By the laws of nature we know that
the glass will break at a future point t when the objects impacts.
Therefore the effect (the glass breaking) can be considered
determined. From such a setup it easily follows why the initial world state is a
sufficient cause for the effect to happen: there is nothing else needed for
the glass to break other than the ball being on a collision trajectory.\\
Now why is the initial world state also a necessary cause for the glass to break?
Surely another object, also on a collision trajectory could have equally caused the glass to break.
This may very well be true for the very abstract effect of “the glass breaking”,
it does not hold however for the effect being the entire resulting world state.
If we account for every last shard of glass, the exact sound waves being emitted from
the collision as well as where the colliding object ended up,
then we have to conclude that there could have only been one previous state the world could have been,
to produce such an outcome (effect). Quoting Perry: “With [traditional determinism] the world [is] a vast machine,
with no loose parts”\autocite[229]{Perry}.
This becomes clearer when comparing one cause lets say it was a green ball breaking the glass to another one
where a red ball was the cause. In the case where the red ball hit the class one would need to ask:
If it was this ball that hit the glass where did the green ball end up in this instance?
As in the common notion of determinism no objects are created nor destroyed one must answer:
Somewhere else! Thereby making the second
situation a completely different effect as the green ball is at a different place.\\
\\
Because of this way determinism is traditionally understood, Ayers distinction between causes that constrain and
ones that don't no longer holds up, thereby rendering the argument ineffective.

\printbibliography

\end{document}
