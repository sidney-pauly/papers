\documentclass[fleqn,14pt]{article}

\usepackage[letterpaper,margin=0.75in]{geometry}

\usepackage{amsmath}
\usepackage{booktabs}
\usepackage{graphicx}
\usepackage{listings}
\usepackage{fancyhdr}
\usepackage{standalone}
\usepackage{float}
\usepackage{hyperref}
\usepackage{biblatex} %Imports biblatex package

% Bibliography
\addbibresource{all.bib}

% \include{data/reaction-time.csv}

\setlength{\parindent}{1.4em}

\pagestyle{fancy}


\begin{document}

\lstset{
  language=Python,
  basicstyle=\small,          % print whole listing small
  keywordstyle=\bfseries,
  identifierstyle=,           % nothing happens
  commentstyle=,              % white comments
  stringstyle=\ttfamily,      % typewriter type for strings
  showstringspaces=false,     % no special string spaces
  numbers=left,
  numberstyle=\tiny,
  numbersep=5pt,
  frame=tb,
}

\title{How do the elements of cinematography affect our interpretation of what is seen in ""Suzhou River"" directed by Ye Lou }
\date{}



\author{Sidney Pauly}

\makeatletter

\let\thetitle\@title
\let\theauthor\@author
\let\thedate\@date
\def\theuoastudentid{52104132}

\makeatother




\fancyhf{}
\fancyhead[L]{Name: \theauthor}
\fancyhead[R]{ID: \theuoastudentid}


% \maketitle

\begin{titlepage}
  \begin{center}
    \Large
    \textbf{\thetitle}
        
    \vspace{0.4cm}
    \large
    FS1508 - Essay 1
        
    \vspace{0.4cm}
    \textbf{\theauthor}\\
    \textbf{\theuoastudentid}

       
    \vfill


    University of Aberdeen\\
    Scotland\\
    UK\\
    \thedate
    \vspace{0.4cm}
    \url{https://github.com/sidney-pauly/papers}
  \end{center}
\end{titlepage}


The cinematography of "Suzhou River" is very unusual by conventional standards. I will argue that
the camera work is not only supporting the story. It succeeds in letting the viewer feel
aimlessness, disorientation, indiscriminateness, and a general loss of identity. Those are all main
motives of the film's storyline. They would however not be transported to the viewer, in the same way,
if it wasn not for the carefully crafted cinematography. 

To show the difference the cinematography is making for "Suzhou River", it is useful to first look
at another film, with similar motives, but more traditional cinematography: Vertigo.\\
Both films tell a story about a man who mistakes one woman for another one.
Mardar mistakes Meimei for Moudan and Scottie mistakes a double for Madeleine.
Both films successfully confuse the viewer about the woman's identity. \\
In Vertigo Scottie, the main character is tasked by a business
man to spy on his wife. At the end of the movie, it turns out that the businessman killed his wife
and swapped her for a doppelgänger, which then staged killing herself to hide the actual murder. 
While Scottie is tricked by the businessman, the viewer is tricked by the director Hitchcock. He
carefully chooses set design, makeup, the setup of the camera and other cinematographic techniques
to hide who is who.\\
"Suzhou River" goes beyond this. While also employing similar techniques to Vertigo, when it comes to
mise en scene aspects, the cinematography achieves something vastly different.\\

In Vertigo the viewer is left out of the story. They are not Scottie. "Suzhou River", by contrast,
the changes perspective between the two man in the story, thus leaving us confused
not only about the identity of Moudan/Meimei, but also about which of the two we are currently following. 
A scene illustrating this very well is the sequence where Mardar is introduced to the story and the perspective
changes for the first time.
\\ It begins with the nameless "main" character, who is also the narrator, looking down onto
a street below. He is waiting for Meimei, his girlfriend, to return home. Once Meimei enters the screen
the camera starts to zoom in on her and tracks for as long as possible, before she is out of frame again,
presumably entering the building. \\
After a brief scene inside, talking to Meimei, the camera returns to the street as the protagonist starts introducing Mardar. 
The camera is now panning, tilting and zooming around the chaotic scene on the street below.
Fixating on each individual for a brief moment, just long enough for the viewer to get invested
in them before losing tracking and following another individual. Seemingly more from a lack of effort in fixating on anything
in particular than the concrete motive of searching for someone. After the camera tracks Moudan, the woman that will later be
mistaken with Meimei, twice, it finds Mardar. \\
What is noteworthy about the sequence, is that between the first and second scene showing the street, we went back in time 
around five years, switched to another character, and left the 1st person perspective. Any first-time viewer of the movie 
would probably not take note of any of this. \\
Usually, movies try to make it easy for the viewer to orient themselves in the setting.
In contrast, "Suzhou River" intentionally breaks this convention. It films two scenes that should be a significant transition,
back to back, in the same way without any change in style. This leaves the viewer disoriented. \\

Besides this intentionally confusing way of compositing, various other cinematographic effects get employed.
One very unusual decision, by conventional standards\cite[p. 36]{Dick}, is to have a lot of scenes shot in the first person.
By doing so it is made hard for the viewer to distance
themselves from what is happening on screen. One scene, which repeats itself from the perspective of both men,
illustrates this very well\footnote{This is also another example where we are mislead, to whom we are following, due To
shots being the same from both perspectives}. In the scene the man stalk Meimei. While she is preparing for her
show the men peek through a curtain, to watch her unobserved. The scenes are shot in first person, which is very uncomfortable
to the viewer as they are, in effect put into the position of invading Meimei's privacy. Shooting in first person means the viewer,
can't distance themselves from the action on screen. In a way they get sucked into the movie. Because of the intentionally confusing
setup, the narrating protagonists, Mardar, and the viewer's identities are all mixed together. There is some debate whether or not 
the viewer becomes the character when something is filmed in first person\cite[p. 216]{Bordwell}, but at the very least, it brings them a lot closer to
the actions taken on-screen. \\

This gets intensified by three other cinematographic decisions: Framing, Focus and distance to the subject.
A lot of scenes are shot extremely close to the characters' faces or their body. This again makes the viewer uncomfortable
as it can feel like an invasion of privacy. A good example for this is when Meimei kisses the
narrating protagonist. As it is filmed in the first person, she comes extremely close to the camera and kisses it directly.\\

Together with the framing and focus this has an additional result: our view is extremely limited. We see very little in these scenes
as the frame is closed in on the subjects, the background is out of focus and they are very close to the camera, thereby simply
blocking everything from view.\\
The effect of this limited view is two-fold: the viewer is denied any own assessment and, to an extent, interpretation of what is happening
on screen. They are forced to only look at what the protagonist looks at. This makes distancing from the events on 
screen even more difficult. It also adds to the disorientation. As the camera is just focusing in on individual details, we never
get any broader overview of anything. There is no visual anchor we can hold on to while watching the scenes. Everything is just
shown for a small duration, before the camera either zooms away, panns away, or loses focus. Denying those visual anchors
makes the viewer literally untethered, aimlessly floating [down the "Suzhou River"].\\

This, in combination with the setting, leads to the general view distance being very limited in all shots.
Either it is done through camera work, by filming towards the ground, against a close wall, etc. or it happens
due to the environmental factors. During the whole movie the city is encased in smock or it rains.
Both limit the view distance drastically. This again denies visual anchors and makes establishing shots practically impossible.
Furthermore, it makes the viewer lost to where exactly they are (and to an extend why they are there). It also
produces the feeling that the run down dirty shanghai, the movie is set in, is endless and without any escape. As
far as we and the characters are concerned it could very well just span the whole globe, allowing no escape\\

Also noteworthy is that there are almost no clean shots in the film.
Every shot is hand held and in a way sloppy. The camera is constantly zooming, tilting, panning or changing
focus. This again leads to disorientation. It also showcases the lack of a clear goal and general aimlessness.
Like described in the shot observing the street, the motivation and energy of the protagonists is not high enough 
to focus on anything for more than mere moments.\\

In traditional cinematography pointing the camera down at something means portraying it as powerless, weak or
otherwise inferior \cite[p. 192]{Bordwell}. Ye Lou points the camera down for a different effect. As mentioned, it 
limits view distance. Another reason has everything to do with the movie being filmed in first-person.\\
Pointing the camera down has the same effect as if the viewer, as a human being, were looking down: it expresses 
a feeling of being depressed.
The lookout on life is not forward or even upwards. It is down as like they had nothing to look forward to.
It could also be interpreted as hiding (i.e. not showing ones face) in effect not confronting ones problems.
This portraits the narrating protagonist's tendency to not stand up to issues
as he lets Mardar just take his girlfried, as he is not asking questions about the jobs he is tasked with (i.e. not facing
up to the harm it might be causing) and to an extend him not starting to search for Meimei after she disappears.\\

During the whole movie the camera is so close to the protagonists it is practically glued onto them. Other movies which employ
a similar camera style often switch it up in the last scenes. There the camera is "set free". Often places of the story are shown again
(without the protagonist in them anymore) or there is a transition into a panoramic view. This is either done 
to portray the burden of the story to be resolved (i.e. the viewer is set free from it) or to leave us with a feeling of emptiness,
as the charecters died, their cause was lost or they moved on.\\
In "Suzhou River" we do not get that luxury. Until the very end the camera keeps being glued
to the protagonists and their problems. We are not set free form the feeling of being aimless, without orientation, lonely,
and confused. We, together with the protagonists are left stuck with these feelings.

Vertigos goal is to stage a mysterious crime story, which for a moment misleads
Scottie and us into believing the supernatural, until the truth reveals itself and everything is resolved. "Suzhou River" does
something different: it tries to make us distressed, aimless and leaves us with a feeling of loss of identity.\\

\printbibliography 

\end{document}