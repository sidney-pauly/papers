\documentclass[14pt]{article}

\usepackage[letterpaper,margin=0.75in]{geometry}

% \usepackage{amsmath}
\usepackage{booktabs}
\usepackage{fancyhdr}
\usepackage{standalone}
\usepackage{float}
\usepackage{hyperref}
% \usepackage{biblatex} %Imports biblatex package
\usepackage{amssymb} % For item list in premise conclusion style

% Bibliography
\usepackage{csquotes}
\usepackage[style=mla]{biblatex}
\addbibresource{all.bib}

% \include{data/reaction-time.csv}

\setlength{\parindent}{1.4em}

\pagestyle{fancy}

\begin{document}


\title{Critical response to Parfit's Theory on Personal Identity}
\date{}



\author{Sidney Pauly}

\makeatletter

\let\thetitle\@title
\let\theauthor\@author
\let\thedate\@date
\def\theuoastudentid{52104132}

\makeatother




\fancyhf{}
\fancyhead[L]{Name: \theauthor}
% \fancyhead[C]{}
\fancyhead[R]{ID: \theuoastudentid}


% \maketitle

\begin{titlepage}
  \begin{center}
    \Large
    \textbf{\thetitle}
        
    \vspace{0.4cm}
    \large
    PH201B - Essay 2
        
    \vspace{0.4cm}
    \textbf{\theauthor}\\
    \textbf{\theuoastudentid}

    \vspace{2cm}
    \textbf{Abstract}

  \end{center}

  In his book \textit{Reasons and persons} Derek Parfit, argues that it is not identity what we should care about when we want to know if we will still
  exist in the future. Instead he proposes, that it is survival of someone with psychological continuity to us, that we should be concerned with. This essay will
  address how that claim might be founded on a too narrowly defined account of personal identity.
  
  \vfill

  \begin{center}

    University of Aberdeen\\
    Scotland\\
    UK\\
    \thedate
    \vspace{0.4cm}
    \url{https://github.com/sidney-pauly/papers}
  \end{center}
\end{titlepage}

Parfit introduces what he calls the R-relation. This relation $R$ means that a particular person at a certain time is psychologically
continuous with some person in the past\autocite[262]{parfit}. Furthermore he defines personal identity ($PI$) to consist of this R-relation plus the relation
$U$, which is the uniqueness criterion (i.e. making the identity claim formal in the sense that there can only be one other
object that the identity relation holds up with). Parfit claims that this makes the identity relation complete and that it can be given by
the formula: $PI = R + U$\autocite[263]{parfit}. Furthermore if it is only uniqueness, that is added to the R-relation and if
this uniqueness claim ($U$) does not add any significant value to PI, than it is the R-relation that we should care about. Parfit supports this view
by highlighting that most reasonable humans would choose to be split in two, for example through their brain halves each being transplanted into a new body, over
dying\autocite[264]{parfit}. This is because it preserves the R-relation even if uniqueness is lost. In concrete terms it fulfills the desire for there to be
someone who shares all their held beliefs, desires and motives. The fact that there are now two people (lefty and righty) does play some role, as it can
lead to practical issues, but as far as Parfit is concerned does not take away from the major objective of survival.\\
\\
A problem with Parfits argument is that he poses personal identity ($PI$) as being completely described by his proposed formula. If one is to believe that
it is only the R-relation (and by extension survival of the sort claimed by Parfit) that matters, uniqueness ($U$) must not only be of little (as Parfit
claims only formal) significance, it also has to be the only relevant component that needs to be added to attain ($PI$). If it can therefore be shown that
personal identity comes apart from the R-relation in any other important way, this poses a problem for Parfit. 
Note that Parfit adds that the R-relation means psychological connectedness and/or psychological continuity with \textbf{the right kind of cause}. This qualifier only means
that not all cases of psychological connectedness are valid if it was caused through unconventional means. As it is limited to the cause, i.e. to
the mechanism through which psychological continuity happens (through ordinary survival opposed through teleportation, being kept alive as a brain in a vat or by
things like simulation). It is not concerned with the contents of the psychology. I.e. if someone is physiologically continuous but their psychology changes over time.\\ 
\\
How can personal identity come apart form psychological continuity then? To explain this it is worth looking at ordinary expressions that people use when they are
talking about their future, like "that wouldn't be me". Such phrases might be used when someone is concerned that what they will become in the future is so
dissimilar to them that it would oppose their core beliefs and goals. They might for example be concerned that some mental illness might send them into a
murderous spree, which, quite understandably, opposes their current moral beliefs. Such core moral beliefs are commonly regarded as being part of someones personal
identity\autocite{sep-plato}. Therefore if someone in the future is to be regarded as the same person they ought to hold moral believes that are at least compatible with their current ones.\\
\\
Parfit might argue that this, does not challenge his proposal as moral beliefs are encompassed in once psychology. Therefore in such a case the person could be regarded
as not being psychological connected even if they are physically the same. As this has to be regarded at least as a possibility it is worth to consider further ways to
extend personal identity.\\
\\
Often personal identity is not seen as something only limited to ones mind, independent of the surrounding. In many cases people see their environment and their
personal relationships as a vital part of who they are. If someone is asked, who they are they cannot answer this question in a direct manner. Rather they have to resort
to relations they have to their environment. Someone might tell you what their job is, that they are the sibling of a person already known to you, etc. It seems to be impossible
to convey information about someones identity, without referring to these relations. If one accepts, like Parfit is suggesting, a reductionist view of the world, the lack
of another way of conveying this information is not due to linguistic limitations. Rather it is because there is nothing about a person besides their physical relation to other
things in the world.\footnote{ If one looks at themselves, there is of course another element to identity, that can be asserted through reflection. Namely that it is themselves
forming the very thought they now have. While this holds for a specific moment in time, it is less clear that it holds for the past as there are various ways our memories, can be distorted or
wrong. There is an argument to be made that even from an inside perspective one needs to resort to environmental identity to know that they are still the same person. Settling
this certainly requires further discussion, but it is at least hard to disprove while holding a reductionist world view }\\ 
\\
This is a problem for Parfits account. If someone splits in two, their relations to the world are disrupted in a meaningful way: there is now a second person
occupying the same relations. While the R-relation would be unaffected, personal identity might be undermined in a way that goes beyond the mere lack of uniqueness ($U$). Crucially
this also disarms Parfits paradox of asking how the survival of two instances of a person can be worse than one in a fundamentally important way. If personal identity
depends on relations, than it is maintained if there is only one person occupying them. In contrast it might be undermined for at least one of the copies, in the case of double occupation.
I.e. because of the disruption of their environmental relationships they become someone else.
An interesting effect of this view is that it is very dependent on the nature of the relations someone has before they split. 
If said double occupation does not pose a threat to someones personal identity, Parfits account still holds.
Crucially though, it is not neccerarrily the case that the R-relation is the only thing that matters. In cases of double occupation, personal identity can be threatened
in ways such that, it comes apart both from the uniqueness requirement as well as the psychological continuity requirement.

\printbibliography

\end{document}
