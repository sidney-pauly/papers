\documentclass[14pt]{article}

\usepackage[letterpaper,margin=0.75in]{geometry}

\usepackage{booktabs}
\usepackage{fancyhdr}
\usepackage{standalone}
\usepackage{float}
\usepackage{hyperref}
\usepackage{amssymb} % For item list in premise conclusion style

% Bibliography
\usepackage{csquotes}
\usepackage[style=mla]{biblatex}
\addbibresource{all.bib}

\setlength{\parindent}{0em}
\pagestyle{fancy}

\begin{document}

\title{Speech in the context recorded and fictional material}
\date{}



\author{Sidney Pauly}
\def\theuoastudentid{52104132}

\makeatletter

\let\thetitle\@title
\let\theauthor\@author
\let\thedate\@date


\makeatother



\fancyhf{}
\fancyhead[L]{Name: \theauthor}
% \fancyhead[C]{}
\fancyhead[R]{ID: \theuoastudentid}

% \maketitle

\begin{titlepage}
  \begin{center}
    \Large
    \textbf{\thetitle}
    \\
    A critical response to Rae Langton's work \textit{Speech Acts and Unspeakable Acts}
    and Jenifer Saul's piece \textit{Pornography, Speech Acts and Context}
        
    \vspace{0.4cm}
    \large
    PH2532 - Essay
        
    \vspace{0.4cm}
    \textbf{\theauthor}\\
    \textbf{\theuoastudentid}

    \vspace{0.9cm}
    \textbf{Abstract}

  \end{center}

  In her essay \textit{Speech Acts and Unspeakable Acts} Rae Langton proposes that pornography is a form of speech act. Furthermore
  she claims that it is plausible to view (at least some) pornographic works as a concrete act of subordination. Jenifer Saul opposes this
  view claiming that it only makes sense to talk about the effects of speech in the given contexts it is perceived in or listened to. This essay will 
  examine what forms of speech appear in the context of recorded or fictional works like pornography. It will then show which form of speech 
  Langton and Saul are talking about in their respective arguments. Finally it will be shown how the claim, that a particular work (e.g. pornography)
  is a particular action in in itself (subordination), can be coherent in spite of Saul's objections.

  \vfill

  \begin{center}

    University of Aberdeen\\
    Scotland\\
    UK\\
    \thedate
    \vspace{0.4cm}
    \url{https://github.com/sidney-pauly/papers}
  \end{center}
\end{titlepage}

Giving an account of speech in the ordinary context does not seem too difficult: speech is information conveyed by a speaker to a listener. Important
in the context of the discussed works is that they are concerned not with speech itself but with speech acts. Soul puts it like this:
"[...] the speech act is that of uttering the sentence (or series of words), not the sentence (or series of words) itself"\autocite[p. 235]{Saul}.
Beyond the act of speech (uttering something) there is the action performed by doing so. I.e. by saying "I would like a coffee" at a cafe, one does
not only say that sentence but is simultaneously also performing the action of ordering a 
coffee\footnote{This is what J.L. Austin coined the "illocutionary act"\autocite{Austin}}.
In ordinary contexts, like ordering said coffee, it is clear that such an action is performed and by whom it is performed. \\ 

In her work, Langton proposes that pornography (and by extend any recorded or fictional work) could be counted as speech. Considering the initial definition,
this makes sense: it conveys some information (i.e. the words written, the images recorded or the sound transmitted) by a
speaker (the creator, director or writer) to a listener (the consumer of the given work). Thereby, it seems to follow that this form of speech can
also be the performing of an action (an illocutionary act)\autocite{Langton}. However, speech of this form seems to be less well defined as ordinary
one:

\begin{enumerate}
  \item[-] There are potentially multiple listeners and sometimes even multiple authors
  \item[-] The work is consumed in multiple contexts and as a result it may be understood in different ways
  \item[-] The content of recorded or fictional work and how it is perceived is not necessarily a direct result of the authors intention
  \item[-] Because of the different contexts the work is being consumed in, different thing might result of it.
  I.e. there are multiple illocutionary acts being performed
\end{enumerate}

It seems clear that the speech act performed through fictional or recorded work (if any) is not singular. Examining this further there seem to be multiple 
ways how speech acts can be performed or carried out through or with fictional works:

\begin{enumerate}
  \item[1.] As already mentioned there is the speech act performed by the producers to some or all recipients of the recording or work
  \item[2.] There is the the speech acts being carried out within the recording or work (i.e. when one character within the recording or fictional work speaks to another)
  \item[3.] There is the speech act carried out when someone shows the recording or work to someone else. 
  By doing so they might want to tell the other person something or convince them of something
  \item[4.] Finally there is the recording or work speaking to the recipients directly (this requires some further explanation) 
\end{enumerate}

These four different ways of speech are illustrated when examining the case of a play being performed for an audience. There is a writer of the play who has the
intention to speak to the audience on a particular matter. They might also want to cause something in the audience by showing them the work (Type 1).\\
Then there is the speech
acts being performed within the play. One of the characters might speak to another one on stage. 
These speech acts might also carry their own illocutionary acts: a character
within the play could equally ask for a coffee and thereby order one (Type 2).\\
Furthermore, there might be someone 
in the audience bringing someone along to the play. They might
do so because they want the other to get a certain impression from the play and learn something from it. 
They might also be trying to perform a completely different action,
like showing their love to the other person by bringing them along (Type 3).\\
Lastly there is the work speaking to the audience directly. In the case of a play this is can be imagined very figuratively: 
If an actor on the stage orders a coffee
within the play they would not only be speaking to another actor on the stage, the speech would equally be directed at the audience (Type 4).\\
\\

In her essay Langton mainly focuses speech of form 1. What she is trying to show is that as pornography is a speech act of this form, it can also be an illocutionary act
(i.e. subordination). The problem with the account, as Saul points out, is that it happens multiple times, once for each of the consumers of the work. Therefore, 
it is not possible to ascribe one concise illocutionary act to all of the speech acts performed. However, some of the examples Saul provides to show this seem to fall into
some of the other categories of speech types outlined. It might be the case that not all of these speech types are relevant in determining the illocutionary act
of a work. \\

Two of her examples fall into category 3:
\begin{enumerate}
  \item The example where an older brother is trying to convince their younger sibling that subordinating women is fine by showing them subordinating pornography. In this
  case the material would have the illocutionary effect Langton ascribes it to have: subordinating women.
  \item The example where subordinating pornography is shown at a philosophy conference as an illustration. In this case the illocutionary act of the material is
  that it confirms the suspicions of the conference's participants, that the porn is subordinating. However, this is not a case where the material has a
  subordinating effect.
\end{enumerate}

To illustrate why this does not necessarily mean that the illocutionary act of pornography might not be either of these performed actions,
it is worth going back to the coffee example:
\\
One could imagine the same setup where Person A order said coffee at a cafe. In addition there are two further people present (B and C). One of them (Person B)
is a regular at the place 
and knows that Person A orders the exact same drink at the exact same time every day. For some reason Person B wants to convince Person C of this fact and that is why they brought them
to the cafe. If Person A now orders the coffee, they are no longer only performing the act of ordering a coffee, they are now also performing the action of proving B's point
to C. \\
This seems to be a similar example to the ones Saul is bringing up. However in this case, it still seems intuitive that saying "I would like a coffee" is
performing the action of ordering one. Similarly, if pornographic material is shown in a different context and thereby is also performing a different action
in that context then it does not necessarily follow that it isn't primarily subordinating. Just like saying "I would like a coffee" is still primarily ordering one.\\

This shows that it might be enough to look at the speech act of form 1. However the problem persists, that this form of speech has
multiple effects, one for each recipient.
It is difficult to justify why one type of these effects should prevail over the others. Langton acknowledges this and proposes some ways of prioritizing these
effects\autocite[p. 311]{Langton}. These ways might hold some merit, however there might be a more effective way of defending Langton's 
argument by focusing on speech of type 4. (i.e. the recorded or fictional work speaking directly to the audience).
This might resolve these problems entirely and also more clearly show how further objections by Soul might not be effective. \\

A further challenge posed by Soul is that recorded work is always dependent on the recipients, and therefore the context it is consumed in.
Soul illustrates this through
an example where people talk to each other through pre written signs. The same sign saying "I do" can be used in different contexts. For example when asked
if one wants a coffee or if one is asked if they committed a murder. The setup seems similar to recorded or fictional works. There is an author of the sign
who may or may not have had a specific intent when writing it. Every time the sing is used (potentially by different people) the performed illocutionary act
might be a different one (for example ordering a coffee or admitting ot murder). Soul implies that just as the sign can be shown in completely different contexts,
a recorded or fictional work could be as well. The recording or work could therefore also perform completely different and diverging illocutionary acts.\\

Here is where focusing on the four types of speech is useful. A sign with just a simple phrase on it
could for example hardly speak directly to the audience (speech of type 4). 
It has to be either used in speech of type 1, 
where the author is trying to convey something to the audience
or in speech of type 3, where it is just a vehicle for someone to perform their potentially unrelated illocutionary act. \\

Fictional and recorded work is more powerful than this. It is capable of speech of type 2: characters within can speak to each other. This means examples 
like the coffee one, are possible within that work. I.e. if a character within the work is saying "I would like a coffee" they are \textbf{primarily} ordering a coffee.
Speech of type 1 is of course not helpful to settle the debate: a movie showing how bad racism is will most likely have instances of racism. Therefore, speech
of type 1 will not tell us much about the illocutionary act performed by the work overall.\\

What is left is speech of type 4: the work directly speaking to the audience. As alluded to, this form of speech needs 
some further explanation, especially as neither Langton nor Soul did not mention anything like it. What does it mean that the work
is speaking directly to the audience?\\
Recordings or a fictional works have their own internal reality and rules. What can happen and does happen within the recorded
work is mostly bound to these rules. How these rules play out is not depended on the external context the work is shown.
"I would like a coffee" will always have the illocutionary effect of ordering a coffee, regardless of
the context the work is shown in.
These rules could be thought of being "self contained" within the work. This also extends
to the author. They could not just decide that saying "I would like a coffee" is the illocutionary act of ordering to shoot someone.
They would have to change the internal rules of the fiction to do so (i.e. make the sentence a code word within the fiction).
All of this means that the work can stand for itself. It is therefore also reasonable to say that it 
can speak for itself independent of the author or the
context it is shown in (which a sign could not).\\
It is important to emphasize that this does not mean that the illocutionary act within a work extends to the external
illocutionary effect of the work. I.e. just because a particular action is always subordinating within the work does not mean that the work itself
is subordinating. It is primarily meant to emphasize that a recorded or fictional work is much more stable in what it is conveying compared to 
things like signs of the sort previously discussed.\\
The work speaking for itself does also not make its illocutionary effects singular.  However, it makes it possible, at least in
principle, that there is a primary illocutionary act being carried out. Furthermore, even if there is no primary effect, the effects
caused could be argued as being independent from the recipients (and the author).\\
\\
Tying all of this together one could therefore argue that by having pornography around, it is speaking to the audience directly (Type 4 speech).
As shown, this speech is neither entirely dependent on what the author tried to convey, the context it is shown in nor the audience it is shown to.
This would make it possible to establish universal illocutionary effects as well as clearly establish which of these are more pronounced. This means
that certain pornography could have the primary (or very pronounced) illocutionary effect of subordination. 
By extension, it would therefore be coherent to claim that some pornography is subordinating.



\printbibliography

\end{document}