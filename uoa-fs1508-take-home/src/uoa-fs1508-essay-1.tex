\documentclass[fleqn,14pt]{article}

\usepackage[letterpaper,margin=0.75in]{geometry}

\usepackage{amsmath}
\usepackage{booktabs}
\usepackage{graphicx}
\usepackage{listings}
\usepackage{fancyhdr}
\usepackage{standalone}
\usepackage{float}
\usepackage{hyperref}
\usepackage{biblatex} %Imports biblatex package

% Bibliography
\addbibresource{all.bib}

% \include{data/reaction-time.csv}

\setlength{\parindent}{1.4em}

\pagestyle{fancy}


\begin{document}

\lstset{
  language=Python,
  basicstyle=\small,          % print whole listing small
  keywordstyle=\bfseries,
  identifierstyle=,           % nothing happens
  commentstyle=,              % white comments
  stringstyle=\ttfamily,      % typewriter type for strings
  showstringspaces=false,     % no special string spaces
  numbers=left,
  numberstyle=\tiny,
  numbersep=5pt,
  frame=tb,
}

\title{Take home exam}
\date{}



\author{Sidney Pauly}
\def\theuoastudentid{52104132}

\makeatletter

\let\thetitle\@title
\let\theauthor\@author
\let\thedate\@date


\makeatother




\fancyhf{}
\fancyhead[L]{Name: \theauthor}
\fancyhead[R]{ID: \theuoastudentid}


% \maketitle

\begin{titlepage}
  \begin{center}
    \Large
    \textbf{\thetitle}
        
    \vspace{0.4cm}
    \large
    FS1508 - Take home exam
        
    \vspace{0.4cm}
    \textbf{\theauthor}\\
    \textbf{\theuoastudentid}

       
    \vfill


    University of Aberdeen\\
    Scotland\\
    UK\\
    \thedate
    \vspace{0.4cm}
    \url{https://github.com/sidney-pauly/papers}
  \end{center}
\end{titlepage}


\section{Question 1.1}
\textit{Vertigo} (Hitchcock, 1958) can be both considered conventional and unconventional in
its editing style. While the movie generally adheres to the rules of continuity editing those rules get purposefully and
skillfully broken to achieve various effects.\\
\\
In classical continuity editing conversation between two characters is usually introduced by having an establishing shot of the location,
followed by a scene where the two characters are both visible on screen facing each other. This establishes to the viewer who is participating in the conversation\cite[p. 180]{Cor}.
During the conversation the camera then usually switches back and forth between the two subjects. This is either done through a shot and counter shot approach: first the camera looks one
character over the shoulder, while they look at their conversation partner followed by the camera then looking over the other characters shoulder while they are looking back at the first
character \cite*[p. 184]{Cor}. Alternatively or in conjunction with the first approach closeup shots can be used
(this is also useful to show characters emotions as a reaction to what is said).
During the entire conversation the 180 rule (dictating the space along a plane connecting the characters and limiting the camera to only one side of said space) is adhered to\cite*[p. 182]{Cor}.\\
\\
Examining scenes in Vertigo will reveal how these rules are adhered to and then broken. A good example is when Midge visits Scottie in the Psychiatry clinic.
The scene starts with an establishing shot of the clinic. It then proceeds to a shot with Scottie and Midge, thus making it clear that they will be partaking in this conversation.
Midge starts out on the right, standing, while Scottie is sitting on a chair to the left. This gives us the initial plane (or axis) for the 180 rule. Midge than starts talking to Scottie.
As we would expect from continuity editing the camera than switches to a reaction shot of Scottie who starts looking at Midge. The camera than switches back and forth
between showing Midge talking and Scottie looking back. While continuing to talk, she slowly starts to get irritated as Scottie is not answering. 
A critical point is reached when Midge realizes that she is having a monologue and that her words might not even get through to Scottie. The editing reflects her
discomfort as the camera subsequently crosses the axis setup for the 180-rule. Breaking the 180-rule in this instance is not an oversight but a clever way to transport Midge's feelings
to the viewer. As the 180-rule is commonly adhered to in most cases the viewer (who doesn't even need to know about the rules existence) is taken aback and gets irritated, as this rule 
gets broken. After the 180 rule is broken the camera again goes back and forth between the two while she talking, but instead of showing Scottie straight on, like it would be done for 
usual conversation, he is filmed diagonally from back and above. It then cuts to a shot which contains Scottie and Midge, who is looking down on him sitting on the chair. This “claims”
the previous shot, that was diagonally down as Midge's perspective. In the next shot the sound of a door can be heard followed by Midge looking up. Finally we see Midge's perspective again,
as she is looking at a nurse, who just entered.\\
\\
Essentially the scene begins with being edited in a style consistent with filming a conversation and ends with as style more
consistent with the subjective view of one character observing something. This supports  the narrative as Midge tries to have a conversation with Scottie but
is left to herself as Scottie seemingly doesn't even notice she's there. While beginning and end are filmed in two distinct styles consistent with continuity editing, in between the
two styles merge into each other, which leads to some conventions to be broken. As alluded to with the 180-rule this isn't a problem but rather helps to convey the uncomfortable feeling
of the scene.


\section{Question 2.3}
\textit{Point break} (Kathryn Bigelow, 1991) uses various techniques to introduce its characters and their motives.
The traits and motives laid out are then later build upon to create suspense and tension.\\
\\
Point break is telling the story of special agent Utah who is conducting an who is conducting an undercover investigation to find out who the bank robbers under the name 
“The ex-presidents” are. As the FBI suspects the robbers to be surfers, the special agent is learning how to surf. 
The leader of “the ex-presidents” is Bodhi. The special agent and the viewer get to know Bodhi, before
it becomes clear that he is part of the bank robbery gang. Later in the movie Utah gets into a conflict with himself
when he needs to decide if he should shoot Bodhi in a chase. On the one hand this would bring down the bank robbers on
the other hand he is sympathetic to Bodhi as they became buddies over the past month. These two conflicting motives of Utah get introduced
through various means before this scene pans out midway through the movie. As such Point break tells a typical Hollywood narrative of a individual battling with
himself \cite[p. 94]{Bordwell} \\
\\
First there is his drive to do exceptional FBI work. The very first scene, which is interlaced into the intro, shows Utah on a shooting range.
A trainer signals the start of a performance test trial. Utah, who is leaning against a car, loads his shotgun and starts shooting at various mannequin paper targets.
After hitting all his shots, he switches to a handgun and proceeds to shoot at the targets. We than see the trainer stopping a the clock, which presumably measured the time it
took Utah to complete the test scenario. Utah looks at him with an expecting expression. The trainer looks down at his notes and states: 
“100 per cent Utah”. The film back to Utah who gives a thumbs up and grins back at him with a happy, innocent and almost child like expression. The scene, conveys all of Utah's initial 
traits and motives to the viewer. He is clearly quite tough, not being repulsed by harder circumstances (It's raining heavily throughout the scene). He seems to have very good shooting
skills and it can be presumed that he is overall a very good agent as he fits into the according character type\cite[p. 257]{Cor} (He seems to be the ambitious academy student, who thinks he
knows what he is doing). He is likely also being perceived as component and highly skilled as he is being played by Keanu Reaves who plays similar roles in films like Matrix (p.258).
At the same time he is presented as inexperienced in life through his facial expression. Through this It's is made clear that he has yet to prove himself.  Also noteworthy is how the
scene is interlaced with the intro, which shows surfing scenes of the main characters, underlined with calm music. Utah's shooting is in odd's with this with the calm music and
interrupts it violently. Through the editing, the calm and beautiful surfing images are juxtaposed with the harsh, violent and dim shooting. This gives a first insight into the
internal conflict that Utah will face later in the movie, where the uptight FBI work attitude and actions are constantly clashing with the surfer lifestyle and Bodhi's spiritual
outlook on life. \\
\\
Over the following scenes Utah arrives in Los Angeles, being assigned to work with the bank robbery team. Overall, the atmosphere is unwelcoming and cold.
It is repeatedly made clear to Utah that he should not expect any respect from any of his co-workers, who think little of him due to not having any field experience.
This peaks in an outburst where he gets into a shouting argument with his FBI partner, as he Is refusing to believe that Utah could do anything to resolve their hardest bank robbery
series by “The ex-presidents”. Utah wins the argument and his partner agrees to give him a shot at solving the case. Up to this moment all the scenes cement Utah's first main motive
of wanting to prove himself and to perform. \\
\\
Later, after being undercover for a while and having learned how to surf Utah gets to know Bodhi who subsequently saves him form an aggressive “Nazi surf gang”. Bodhi
states that he dislikes them, not due to the fact that they might commit felonies (a motive that Utah as an FBI agent, is mainly concerned with) but because they have the wrong attitude 
towards surfing. For him surfing has a spiritual side and he states: “You lose yourself in the waves and you find yourself in the waves”. Before Utah leaves Bodhi invites him to his 
house party later that evening.\\
\\
Later at the party Utah and Bodhi are gathered around a campfire with other people. In this scene Utah's second main motive, him sympathizing with Bodhi as well as his way of life,
gets introduced. Bodhi sit's at the campfire resting in the arms of a woman massaging his back. During the scene the camera films Bodhi and only Bodhi only through the flames of the campfire.
This can be understood as in the expression of someone having flames in their eyes, meaning they are callous to do great things but are at the same time reckless to
what other consequences their behavior might entail. It can also be understood as a symbol for the looming danger that will come over Utah due to Bodhi. Gathered around
the campfire different suffers tell their surf stories and their attitude towards going into the waves. Some of the surfers are of the opinion that surfing big waves would be
the ultimate thrill. They argue that due to their danger they mean ultimate commitment. They also state that they would risk their life's to surf to seek that thrill.
Finally the conversation shifts towards what the biggest wave is and Bodhi get's to tell his theory about the “50 year wave” which he believes to appear next year.
In a calm and collected manner (in contrast to the other surfer's exited attitude) he again expresses his spiritual attitude towards life and then states that he is willing 
“to pay the ultimate prize” (i.e. death) to catch those waves. In contrast to the previous speakers he seems generous with that attitude, thereby affirming to Utah and the viewer 
that he has a death wish and is ready to go to great length to fulfil that goal. During the whole scene Utah is standing at the fire. Noteworthy is how he is half illuminated,
one side of his face in complete darkness the other side in bright orange from the campfire. This symbolizes how he is torn between the two previously mentioned motives.\\
\\
Afterwards Utah gets a taste of Bodhi's lifestyle as they go surfing during the night (which is quite dangerous and thrilling) and ends up spending the night
sleeping on the beach together with a women he is falling in love with. This leads to him waking up late and arriving late to a mission in the morning.
This makes it clear to the viewer that the two lifestyles are not compatible and that his FBI work is starting to take hits due to the same.\\
\\
All of this setup culminates in a scene where Utah chases the Bodhi after he made another bank robbery with “The ex-presidents”.
During the chase Utah falls and injures his leg dropping him down to the ground. He manages to get his gun out and point it at Bodhi,
who during the entire scene is masked so he doesn't know that Bodhi is actually part of the gang. Utah hesitates to pull the trigger as
he Is reluctant to kill. As a result he stares the man climbing the fence directly into the eyes realizing that it is in fact Bothy.
They look at each other for a few seconds (showing that they feel a strong connection between each other).
Utah now needs to make the decision weather to shot (ending “The ex-presidents” and proving himself within the FBI) or to let Bodhi go.
He ends up choosing the later, discharging is gun into the air with anger.\\
\\
Overall point break various ways of letting the viewer look at the inside processes of its characters. It employs all the different kinds of techniques available in film and spends a great amount of its runtime to introduce the character motives in a believable way. This then culminates in the second half of the movie where this conflict can then be fought out such that the viewer fully understands Utah's dichotomy.



\printbibliography 

\end{document}