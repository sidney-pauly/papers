\documentclass[14pt]{article}

\usepackage[letterpaper,margin=0.75in]{geometry}

\usepackage{booktabs}
\usepackage{fancyhdr}
\usepackage{standalone}
\usepackage{float}
\usepackage{hyperref}
\usepackage{amssymb} % For item list in premise conclusion style

% Bibliography
\usepackage{csquotes}
\usepackage[style=mla]{biblatex}
\addbibresource{all.bib}

\setlength{\parindent}{0em}
\pagestyle{fancy}

\begin{document}

\title{Can we defeat scepticism about the external world?}
\date{}



\author{Sidney Pauly}

\makeatletter

\let\thetitle\@title
\let\theauthor\@author
\let\thedate\@date
\def\theuoastudentid{52104132}

\makeatother



\fancyhf{}
\fancyhead[L]{Name: \theauthor}
% \fancyhead[C]{}
\fancyhead[R]{ID: \theuoastudentid}

% \maketitle

\begin{titlepage}
  \begin{center}
    \Large
    \textbf{\thetitle}
        
    \vspace{0.4cm}
    \large
    PH2540 - Essay
        
    \vspace{0.4cm}
    \textbf{\theauthor}\\
    \textbf{\theuoastudentid}


  \end{center}


  \vfill

  \begin{center}

    University of Aberdeen\\
    Scotland\\
    UK\\
    \thedate
    \vspace{0.4cm}
    \url{https://github.com/sidney-pauly/papers}
  \end{center}
\end{titlepage}

This essay will argue that while scepticism about the real world cannot be defeated it
might be less powerful in its consequences than its claimed to be. 
% First some skeptic arguments
% will be examined as well as some possible responses. 
% Then a closer look at skeptic scenarios,
% like the brain in a vat case or Descartes Daemon will be examined
\\

Skeptic arguments about our knowledge of the real world can be made in many ways. For a skeptic argument to
be successful it needs to show that at least one requirement for knowledge about the real world is always undermined.
It can for example be done by questioning the certainty of ones beliefs\autocite[p. 135]{lemos_2007}:

\begin{itemize}
  \item[1.] It is impossible to be certain about the real world
  \item[2.] To know something, one has to be certain about it
  \item[$\therefore$] it is impossible to know about the real world
\end{itemize}

This argument only works if one cannot be certain about the real world. However that seems to be the case as there are
multiple different kinds of doubts about how we got to our beliefs and if they are, in fact, reliable. In his famous
works Descartes introduces some of these doubts. We for example all had situations where our senses deceived us and
we later found out what we thought we saw wasn't in fact that way. Descartes points out that if our senses deceive
us sometimes they could deceive us all the time, thus we might actually perceiving the real world\autocite*[p. 194]{descartes_1985}.
Another way to doubt reality that Descartes brings in is that we could always be in a dream state, i.e. just imagining the
world around us: "The second reason is that in our sleep we regularly seem to
have sensory perception of, or to imagine, countless things which do not
exist anywhere; and if our doubts are on the scale just outlined, there
seem to be no marks by means of which we can with certainty distinguish
being asleep from being awake"\autocite*[p. 194]{descartes_1985}. In whatever way these doubts are brought about, they certainly
bring into question how certain we can be about our perception. \\

Within epistemology (and in ordinary use) there are more permissible accounts of knowledge which put in lower requirements for someone
to have knowledge. These
more permissible accounts of knowledge are mostly lowering how certain one has to be about a belief for it to count as knowledge.
While under these theories knowledge about the real world might not be undermined, they are equally not giving high
certainty for our beliefs to be true, thus one will still remain skeptic about the real world even if "knowing" about its
existence.\\

Considering this, in order to dismiss skeptic arguments one has to show that certainty about ones beliefs in the real world is possible
and in fact attained. Doing so seems difficult considering powerful skeptic "what if" scenarios, which supposedly bring about
circumstances in which no real world exists. Two of these scenarios will be examined in this essay:\\

\textbf{Brain in a vat/Matrix/Simulation}: A modern skeptic argument is the brain in a vat scenario. The proposition is that instead of us perceiving actual reality we are just 
a brain in a bath of nutrients, with all the peripheral neurons (neurons handling incoming perception and outgoing motor signals) connected
to a computer. This computer is producing all the right electrical signals to make us believe we are living within reality. The scenario is quite
similar to the well known film "The Matrix".\\
The simulation scenario is essentially the same, just with the brain itself also being simulated in addition to the sensory inputs.\\

\textbf{Descartes evil daemon}: A famous such scenario is Descartes' daemon. Descartes proposes that we cannot be certain of the real world as all our sensory
input could be fabricated by an evil daemon replicating our worldly experiences. I.e. instead of us actually having hands that we
can see and move, the daemon is just fabricating all the sensory inputs required for us to form that belief.

% \subsection{Boltzmann brain}

% A less known but very powerful skeptic scenario is the Boltzmann brain. Boltzmann scenario is that we could just be a brain that randomly
% assembled within space, with all the right memories and believes and then deconstructs immediately after. Boltzmann holds that this scenario
% ought to be more probable than the entire universe composing in exactly the right manner for us to come into actual existence. More relevant to
% the epistemic discussion however is that in this scenario, we would have all the believes about the real world without even being able to rely
% on our past memories or even on the fact that we are existing in a continuous manner at all (i.e. thinking for an extended period of time, rather
% then just existing in the one instance the particles randomly assembled).

Taking these skeptic scenarios one can easily form various arguments why we cannot have knowledge about the real world. A very strong one (as it's
not increasing the requirements for knowledge) is the argument from ignorance\autocite[p. 140]{lemos_2007}:

\begin{itemize}
  \item[1.] One can only know about the real world, iif they know that all contradictory scenarios (like the ones above) are false
  \item[2.] It is impossible to know that contradictory scenarios (like the ones laid out above) are false
  \item[$\therefore$] it is impossible to know about the real world
\end{itemize}

It might be possible to escape the argument by arguing against 1. in so far as that knowledge does not require being able
to disprove all the other scenarios. Again this seems to not to be sufficient as it only recovers the usage of knowledge
in a weakened way. Skepticism about the world would still remain.\\

Therefore a more powerful approach would be to prove that in fact the contrary scenarios do not contradict knowledge about the real world.
A way this can be done is by looking closer at how "real world" is being used in the skeptic scenarios. It seems that in the sceptic scenarios
there is always a different reality under which rules we operate. Something like a simulated world or a fabricated world in the case of Descartes
daemon. Next or around that reality there then is the other reality which is supposedly the "real" one (i.e. the one where the simulation runs, the
reality of the daemon, etc.). One ought to ask what does make those realities more real than the other? \\

There seem some answers to this like:
\begin{enumerate}
  \item As one of the realities is fabricated or faked, its less real
  \item The "outer" one is more real as it's more fundamental, if one takes away the outer one the inner ceases to exist but not vice versa. 
\end{enumerate}

However there are also arguments for the contrary; after all when we ordinarily talk about reality we mean the construct that is perceivable by us.
One could claim that  if we talk about reality, that we are in fact talking about what we perceive by our senses regardless
of the mechanism these senses came about. This point can be examined further by asking how a satisfactory reality has to look like. If one subscribes
to a physicalist view, they think of reality consisting of fundamental particles and forces. However one can still be skeptic about how those things
are fundamental. They could very well just come about, as part of a simulation in a higher reality. In that case we would actually perceive reality
directly it just might be part of something bigger. In that case would it be right to claim it less real, because there is a more fundamental level?\\

More important than if the consideration which reality is more "real" (which might ultimately be a semantic one), are the consequences the sceptic argument
actually imply. Take Descartes example of one not knowing if they have hands. This builds on the sceptic argument above. The conclusion one is supposed to
draw is that because one cannot be certain about the how reality is formed one cannot be certain about the objects within. Initially the argument seems
plausible, if all our sensory inputs are deceptions then our beliefs about our hands might equally be false.
However this seems to be based on the assumption that the belief about having hands is a claim that we have hands in the "outer" reality. However that seems
to be a very specific way of specifying that belief.\\
To explore this in more depth let's consider a way that this sentence could be understood: The sentence could be a claim that there is a object in relation
to me, that I can manipulate in a certain way (i.e. grab objects, etc.) that causes a certain visual perception (a meaty object with five meaty extensions)
and that has certain other characteristics we would ordinarily consider a hand to have.
If the sentence "I know that I have hands" is to be taken like that then the brain in a vat case is not case where I don't have hands. All those
descriptions still apply to the hands I have within the "fake" reality, thus I would know that I have hands.\\

A possible counter argument from a skeptic standpoint could be that our belief about the simulated hands is in fact not knowledge as there is trickery going on.
In other words the object might show all the characteristics of hands, but underneath it's just "smoke and mirrors". This can be resolved by pointing out that
our perception might be deceived, but the connections we form between them aren't. This can be illustrated through the daemon example: the daemon is said to be all
powerful in their deceit. However for the deception to be convincing it has to work in just the right way. Meaning that if one sees a ball within their perception,
they can walk up to it, pick it up, throw it and it will fly in a trajectory described by newtons mechanics. The daemon, cannot part form this behavior, without it becoming
obvious to us that something is wrong with reality. As we can clearly see from our day to day life, reality does not sometimes just break down, on the contrary, it
seems to follow very exact rules to high precision. Going back to the hands, it is quite easy to show, even when deceived by a daemon, that the hand are not just
"smoke and mirrors", by e.g. just taking an x-ray of them. The daemon will have to give us the perception of a correct x-ray image of the hand produced by convenable
physics, otherwise their deception is no longer deceiving. In a way we would control the daemon more than it controls us, as it would be the one needing to go through
all the right images and sensations that we expect. All of this to say, that even in the case reality is faked, this would hardly matter as it would still
work exactly the same way, down to the most precise measurements we can take, otherwise would already have broken down.\\

Overall skeptic arguments, while seeming powerful, might have less of an impact than Initially thought. Even if the skeptic scenarios are true it is still possible
to draw links between ones perceptions and thus form a firm knowledge of objects, as their behavior isn't affected by how that reality came about.

\printbibliography

\end{document}
